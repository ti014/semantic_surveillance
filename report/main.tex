% Semantic Surveillance PPE Monitoring - Slide Deck (Beamer)
% Vietnamese có dấu + English technical terms
\documentclass[aspectratio=169]{beamer}
\usetheme{Madrid}

\usepackage[utf8]{inputenc}
\usepackage[T5]{fontenc}
\usepackage[vietnamese]{babel}
\usepackage{graphicx}
\usepackage{booktabs}
\usepackage{hyperref}

\title{Semantic Surveillance \\ Hệ thống giám sát PPE thông minh}
\author{Phuong Phan Nguyen Mai \and Hoang Ngo Quoc}
\date{\today}

\begin{document}

%------------------------------------------------------------------
\begin{frame}
  \titlepage
  \begin{itemize}
    \item Mục tiêu: Giám sát tuân thủ PPE trên công trường xây dựng.
    \item Trọng tâm hiện tại: YOLOv8n đã huấn luyện (phát hiện PPE chính).
    \item Hướng tương lai: YOLO-World để mở rộng nhanh mà không cần gán nhãn.
    \item PPE (Personal Protective Equipment): trang bị bảo hộ cá nhân như mũ, áo phản quang, găng, giày, kính.
  \end{itemize}
\end{frame}

%------------------------------------------------------------------
\begin{frame}{Vấn đề \& Yêu cầu}
  \begin{itemize}
    \item Rủi ro mất an toàn từ việc thiếu PPE; cần cảnh báo nhanh, chính xác.
    \item Thách thức: nhiều loại PPE, môi trường nhiễu, góc quay phức tạp.
    \item Ràng buộc triển khai: thời gian suy luận thấp, dễ mở rộng PPE mới, hạn chế labeling thủ công.
  \end{itemize}
\end{frame}

%------------------------------------------------------------------
\begin{frame}{Giới thiệu YOLO}
  \begin{itemize}
    \item YOLO là mô hình phát hiện đối tượng theo thời gian thực, xử lý ảnh theo một lần suy luận.
    \item Ưu điểm: tốc độ cao, dễ triển khai trên camera/video.
    \item Hạn chế: phụ thuộc chất lượng dữ liệu huấn luyện và điều kiện môi trường.
  \end{itemize}
\end{frame}

%------------------------------------------------------------------
\begin{frame}{Các phiên bản liên quan trong dự án}
  \begin{itemize}
    \item YOLOv8n: bản nhẹ, nhanh, phù hợp chạy thời gian thực; là trọng tâm hiện tại.

    \item YOLO-World: hướng tương lai, nhận diện theo mô tả văn bản, không cần gán nhãn lại.
  \end{itemize}
\end{frame}

%------------------------------------------------------------------
\begin{frame}{Giải pháp tổng thể}
  \begin{itemize}
    \item Trọng tâm hiện tại: YOLOv8n đã huấn luyện để phát hiện người và PPE (mũ, áo phản quang, giày, găng, kính).
    \item Kiến trúc tách lớp: Detector -- Ghép thiết bị -- Kiểm tra quy tắc -- Ghi log/Hiển thị.
    \item Cấu hình bằng YAML: ngưỡng tin cậy, quy tắc bắt buộc, ánh xạ nhãn.
    \item Hướng tương lai: YOLO-World giúp nhận diện PPE mới chỉ bằng mô tả văn bản, không cần gán nhãn.
  \end{itemize}
\end{frame}

%------------------------------------------------------------------
\begin{frame}{Kiến trúc thành phần (core/)}
  \begin{itemize}
    \item \texttt{detector.py}: YOLOv8n là detector chính; YOLO-World chỉ dùng cho nghiên cứu mở rộng.
    \item \texttt{ppe\_checker.py}: Ghép thiết bị theo vùng cơ thể + kiểm tra đủ PPE bắt buộc.
    \item \texttt{violation\_logger.py}: lưu ảnh vi phạm, metadata dạng JSONL, thống kê theo loại thiếu.
    \item \texttt{visualizer.py}: vẽ khung, nhãn an toàn/vi phạm, hiển thị thống kê.
    \item \texttt{utils/helpers.py}: đọc YAML, hàm hỗ trợ hình học.
  \end{itemize}
\end{frame}

%------------------------------------------------------------------
\begin{frame}{Cấu hình chính (YAML)}
  \begin{itemize}
    \item \textbf{YOLOv8n (chính)}: \texttt{config\_yolov8\_production.yaml}
      \begin{itemize}
        \item Ngưỡng tin cậy 0.50, IoU 0.3 để giảm nhầm lẫn.
        \item Ánh xạ nhãn (helmet$\rightarrow$hard\_hat, vest$\rightarrow$safety\_vest,...).
        \item Quy tắc bắt buộc: ít nhất phải có mũ bảo hộ (có thể mở rộng thêm áo, giày,...).
      \end{itemize}
    \item \textbf{YOLO-World (tương lai)}: \texttt{config\_yoloworld\_future.yaml}
      \begin{itemize}
        \item Dùng mô tả văn bản để nhận diện PPE mới.
        \item Ngưỡng tin cậy thấp hơn để không bỏ sót khi zero-shot.
      \end{itemize}
  \end{itemize}
\end{frame}

%------------------------------------------------------------------
\begin{frame}{Workflow suy luận (main.py)}
  \begin{enumerate}
    \item Load config, khởi tạo detector (YOLOv8 hoặc YOLO-World), \textbf{PPEChecker}, \textbf{ViolationLogger}, \textbf{Visualizer}.
    \item Đọc khung hình (video/camera), resize nếu cần.
    \item Detector: predict $\rightarrow$ \texttt{get\_detections} trả về \{person, hard\_hat, ...\}.
    \item PPEChecker: normalize tên, associate thiết bị với từng person (vùng head/torso/legs, IoE), validate required items.
    \item Logger: lưu crop người vi phạm + JSONL (timestamp, bbox, missing).
    \item Visualizer: vẽ thiết bị + nhãn SAFE/VIOLATION, overlay FPS/thống kê.
  \end{enumerate}
\end{frame}

%------------------------------------------------------------------
\begin{frame}{Lý do chọn kiến trúc này}
  \begin{itemize}
    \item \textbf{Tập trung YOLOv8n}: độ chính xác tốt, tốc độ cao, phù hợp demo và triển khai thực tế.
    \item \textbf{Tách lớp rõ ràng}: thay đổi quy tắc hoặc cách ghép thiết bị không ảnh hưởng detector.
    \item \textbf{Cấu hình hóa}: chỉnh ngưỡng và quy tắc bằng YAML, không cần sửa mã nguồn.
    \item \textbf{Ghép theo vùng cơ thể}: giảm nhầm lẫn khi nhiều người đứng gần nhau.
    \item \textbf{YOLO-World là hướng tương lai}: giảm chi phí gán nhãn khi cần PPE mới.
  \end{itemize}
\end{frame}

%------------------------------------------------------------------
\begin{frame}{Cách làm chi tiết}
  \begin{itemize}
    \item \textbf{Phát hiện (YOLOv8n)}: nhận diện người và PPE, sau đó chuẩn hóa nhãn theo mapping.
    \item \textbf{Ghép thiết bị}: chia vùng đầu/ngực/chân theo tỉ lệ; mũ bảo hộ bắt buộc ở vùng đầu.
    \item \textbf{Kiểm tra quy tắc}: danh sách PPE bắt buộc; thiếu gì thì đánh vi phạm.
    \item \textbf{Ghi log và hiển thị}: lưu ảnh vi phạm + JSONL; hiển thị khung và thống kê.
    \item \textbf{Tương lai (YOLO-World)}: nhận diện bằng mô tả văn bản khi bổ sung PPE mới.
  \end{itemize}
\end{frame}

%------------------------------------------------------------------
\begin{frame}{Ưu điểm \& Nhược điểm}
  \begin{itemize}
    \item \textbf{Ưu điểm (YOLOv8n)}:
    \begin{itemize}
      \item Tốc độ cao, ổn định; phù hợp chạy thời gian thực.
      \item Dễ kiểm soát sai số nhờ ngưỡng tin cậy và IoU rõ ràng.
      \item Kết quả nhất quán trên các PPE đã huấn luyện.
    \end{itemize}
    \item \textbf{Nhược điểm}:
    \begin{itemize}
      \item Cần dữ liệu gán nhãn khi thêm PPE mới.
      \item Khó thích ứng nếu môi trường quá khác dữ liệu huấn luyện.
      \item Chưa có theo dõi đối tượng nên thống kê có thể bị lặp.
    \end{itemize}
    \item \textbf{Ưu điểm tương lai (YOLO-World)}: mở rộng nhanh, ít phụ thuộc dữ liệu gán nhãn.
  \end{itemize}
\end{frame}

%------------------------------------------------------------------
\begin{frame}{Cân bằng kỹ thuật}
  \begin{itemize}
    \item \textbf{Tốc độ vs độ chính xác}: ưu tiên YOLOv8n để đảm bảo thời gian thực.
    \item \textbf{Chính xác vs bao phủ}: tăng ngưỡng để giảm báo sai; hạ ngưỡng để không bỏ sót.
    \item \textbf{Ghép nghiêm ngặt vs linh hoạt}: nghiêm ngặt giúp giảm nhầm, linh hoạt giúp chịu được góc quay khó.
    \item \textbf{Ghi log chi tiết vs hiệu năng}: lưu ảnh vi phạm tăng chi phí I/O; có thể tắt khi cần tốc độ.
    \item \textbf{Tương lai}: YOLO-World đánh đổi tốc độ và độ ổn định để lấy khả năng mở rộng nhanh.
  \end{itemize}
\end{frame}

%------------------------------------------------------------------
\begin{frame}{Association logic (EquipmentAssociator)}
  \begin{itemize}
    \item Chia box người thành region: head, torso, legs (tỉ lệ từ config).
    \item Hard hat bắt buộc check vùng head (strict IoE $>$ 0.3). Thiết bị khác có thể relaxed nếu \texttt{strict\_regions=false}.
    \item Dùng IoE (Intersection over Equipment area) để giảm nhầm lẫn khi box nhỏ.
  \end{itemize}
\end{frame}

%------------------------------------------------------------------
\begin{frame}{Ghép thiết bị và kiểm tra tuân thủ}
  \begin{itemize}
    \item Bước 1: Tách tất cả box \textbf{person} và các box PPE từ kết quả phát hiện.
    \item Bước 2: Với mỗi người, tạo vùng \textbf{đầu/ngực/chân} theo tỉ lệ trong cấu hình.
    \item Tỉ lệ mặc định: đầu 30\%, ngực 50\% (từ 30\% đến 80\%), chân 30\% (từ 70\% đến 100\%).
    \item Bước 3: Ghép PPE vào người bằng IoE; mũ bảo hộ phải nằm vùng đầu, áo nằm vùng ngực, giày nằm vùng chân.
    \item Bước 4: So sánh với danh sách PPE bắt buộc; thiếu gì thì gắn nhãn \textbf{vi phạm}.
    \item Kết quả: mỗi người có trạng thái \textbf{an toàn/vi phạm} kèm danh sách PPE thiếu.
  \end{itemize}
\end{frame}

%------------------------------------------------------------------
\begin{frame}{Huấn luyện YOLOv8n (train\_ppe.py)}
  \begin{itemize}
    \item Dữ liệu: Construction-PPE (\texttt{datasets/construction-ppe/data.yaml}).
    \item Mục tiêu: học đặc trưng PPE phổ biến (mũ, áo phản quang, giày, găng, kính).
    \item Tham số: ảnh 640, 50 epochs (GPU) hoặc 20 epochs (CPU), batch 16/8.
    \item Kết quả: \texttt{runs/detect/ppe\_detector/weights/best.pt} dùng cho chạy thực tế.
  \end{itemize}
\end{frame}

%------------------------------------------------------------------
\begin{frame}{Lưới mẫu dữ liệu (image2--image7)}
  \begin{center}
    \begin{minipage}{0.30\linewidth}
      \includegraphics[width=\linewidth]{images/image2.jpg}
    \end{minipage}
    \begin{minipage}{0.30\linewidth}
      \includegraphics[width=\linewidth]{images/image3.jpeg}
    \end{minipage}
    \begin{minipage}{0.30\linewidth}
      \includegraphics[width=\linewidth]{images/image4.jpg}
    \end{minipage}
    
    \vspace{0.15cm}
    
    \begin{minipage}{0.30\linewidth}
      \includegraphics[width=\linewidth]{images/image5.jpeg}
    \end{minipage}
    \begin{minipage}{0.30\linewidth}
      \IfFileExists{images/image6.jpg}{\includegraphics[width=\linewidth]{images/image6.jpg}}{
        \IfFileExists{images/image6.jpeg}{\includegraphics[width=\linewidth]{images/image6.jpeg}}{
          \IfFileExists{images/image6.png}{\includegraphics[width=\linewidth]{images/image6.png}}{}
        }
      }
    \end{minipage}
    \begin{minipage}{0.30\linewidth}
      \includegraphics[width=\linewidth]{images/image7.jpg}
    \end{minipage}
  \end{center}
  \vspace{0.1cm}
  \begin{itemize}
    \item Tổng hợp nhanh bối cảnh, ánh sáng, mật độ người và mức độ che khuất.
    \item Dùng để đánh giá trực quan tính đa dạng dữ liệu huấn luyện.
  \end{itemize}
\end{frame}

%------------------------------------------------------------------
\begin{frame}{Trực quan nhãn (labels)}
  \begin{center}
    \includegraphics[width=0.68\linewidth]{images/labels.jpg}
  \end{center}
  \vspace{0.2cm}
  \begin{itemize}
    \item Minh hoạ các lớp PPE và phân bố nhãn.
    \item Dùng để kiểm tra dữ liệu có đủ đại diện cho các loại PPE hay không.
  \end{itemize}
\end{frame}

%------------------------------------------------------------------
\begin{frame}{Batch huấn luyện (train batch)}
  \begin{center}
    \includegraphics[width=0.7\linewidth]{images/train_batch0.jpg}
  \end{center}
  \vspace{0.2cm}
  \begin{itemize}
    \item Ảnh batch huấn luyện sau khi augment (biến đổi dữ liệu).
    \item Giúp tăng đa dạng dữ liệu, cải thiện khả năng tổng quát.
  \end{itemize}
\end{frame}

%------------------------------------------------------------------
\begin{frame}{Đường cong Precision-Recall}
  \begin{center}
    \includegraphics[width=0.7\linewidth]{images/BoxPR_curve.png}
  \end{center}
  \vspace{0.2cm}
  \begin{itemize}
    \item Thể hiện cân bằng giữa độ chính xác và khả năng bắt đủ đối tượng.
    \item Dùng để chọn ngưỡng tin cậy phù hợp khi triển khai.
  \end{itemize}
\end{frame}

%------------------------------------------------------------------
\begin{frame}{Đường cong F1-score}
  \begin{center}
    \includegraphics[width=0.7\linewidth]{images/BoxF1_curve.png}
  \end{center}
  \vspace{0.2cm}
  \begin{itemize}
    \item F1-score là trung bình điều hoà giữa Precision và Recall.
    \item Đỉnh đường cong gợi ý ngưỡng tối ưu cho cân bằng sai/đúng.
  \end{itemize}
\end{frame}

%------------------------------------------------------------------
\begin{frame}{Đường cong Precision}
  \begin{center}
    \includegraphics[width=0.7\linewidth]{images/BoxP_curve.png}
  \end{center}
  \vspace{0.2cm}
  \begin{itemize}
    \item Precision cao giúp giảm báo sai (false positive).
    \item Quan trọng với cảnh báo an toàn để tránh làm phiền.
  \end{itemize}
\end{frame}

%------------------------------------------------------------------
\begin{frame}{Đường cong Recall}
  \begin{center}
    \includegraphics[width=0.7\linewidth]{images/BoxR_curve.png}
  \end{center}
  \vspace{0.2cm}
  \begin{itemize}
    \item Recall cao giúp giảm bỏ sót (false negative).
    \item Quan trọng để không bỏ qua công nhân vi phạm PPE.
  \end{itemize}
\end{frame}

%------------------------------------------------------------------
\begin{frame}{Ma trận nhầm lẫn}
  \begin{center}
    \includegraphics[width=0.7\linewidth]{images/confusion_matrix.png}
  \end{center}
  \vspace{0.2cm}
  \begin{itemize}
    \item Cho thấy lớp nào hay bị nhầm lẫn với lớp khác.
    \item Dùng để định hướng cải thiện dữ liệu hoặc thêm rule kiểm tra.
  \end{itemize}
\end{frame}

%------------------------------------------------------------------
\begin{frame}{So sánh nhãn và dự đoán (validation)}
  \begin{center}
    \includegraphics[width=0.48\linewidth]{images/val_batch0_labels.jpg}
    \includegraphics[width=0.48\linewidth]{images/val_batch0_pred.jpg}
  \end{center}
  \vspace{0.2cm}
  \begin{itemize}
    \item Trái: nhãn gốc. Phải: dự đoán của mô hình.
    \item Dùng để kiểm tra trực quan chất lượng phát hiện.
  \end{itemize}
\end{frame}

%------------------------------------------------------------------
\begin{frame}{Ma trận nhầm lẫn (chuẩn hoá)}
  \begin{center}
    \includegraphics[width=0.7\linewidth]{images/confusion_matrix_normalized.png}
  \end{center}
  \vspace{0.2cm}
  \begin{itemize}
    \item Thể hiện tỉ lệ nhầm lẫn theo phần trăm giữa các lớp.
    \item Dễ quan sát lớp nào cần bổ sung dữ liệu.
  \end{itemize}
\end{frame}

%------------------------------------------------------------------
\begin{frame}{So sánh nhãn và dự đoán (validation batch 1)}
  \begin{center}
    \includegraphics[width=0.48\linewidth]{images/val_batch1_labels.jpg}
    \includegraphics[width=0.48\linewidth]{images/val_batch1_pred.jpg}
  \end{center}
  \vspace{0.2cm}
  \begin{itemize}
    \item So sánh trực quan giữa nhãn gốc và kết quả dự đoán.
    \item Dùng để đánh giá lỗi thiếu hoặc thừa bounding box.
  \end{itemize}
\end{frame}

%------------------------------------------------------------------
\begin{frame}{So sánh nhãn và dự đoán (validation batch 2)}
  \begin{center}
    \includegraphics[width=0.48\linewidth]{images/val_batch2_labels.jpg}
    \includegraphics[width=0.48\linewidth]{images/val_batch2_pred.jpg}
  \end{center}
  \vspace{0.2cm}
  \begin{itemize}
    \item Trường hợp nhiều đối tượng và vị trí chồng lấn.
    \item Giúp kiểm tra độ ổn định khi bối cảnh phức tạp.
  \end{itemize}
\end{frame}

%------------------------------------------------------------------
\begin{frame}{Demo sử dụng}
  \textbf{Chạy YOLOv8n trên video/camera:}
  \begin{itemize}
    \item \texttt{python main.py --config config/config\_yolov8\_production.yaml --source bao-ho-lao-dong.mp4}
    \item Lưu video: \texttt{--save-path output/demo.mp4}
  \end{itemize}
  \textbf{Hướng tương lai (YOLO-World):}
  \begin{itemize}
    \item \texttt{python scripts/test\_images.py --config config/config\_yoloworld\_future.yaml --source images\_test --output output/future\_test}
    \item Dùng mô tả văn bản để thử PPE mới, không cần gán nhãn.
  \end{itemize}
\end{frame}

%------------------------------------------------------------------
\begin{frame}{Quy trình YOLOv8n (hiện tại)}
  \begin{center}
    \includegraphics[width=1.0\linewidth]{images/luong_yolov8n.png}
  \end{center}
  \vspace{0.2cm}
  \begin{itemize}
    \item Luồng xử lý: dữ liệu huấn luyện $\rightarrow$ mô hình YOLOv8n $\rightarrow$ phát hiện PPE.
    \item Phù hợp chạy thời gian thực với tốc độ cao.
  \end{itemize}
\end{frame}

%------------------------------------------------------------------
\begin{frame}{Quy trình YOLO-World (tương lai)}
  \begin{center}
    \includegraphics[width=1.0\linewidth]{images/luong_yolov8s_world.png}
  \end{center}
  \vspace{0.2cm}
  \begin{itemize}
    \item Luồng xử lý: mô tả văn bản $\rightarrow$ nhận diện PPE mới mà không cần gán nhãn lại.
    \item Tăng khả năng mở rộng khi có thiết bị bảo hộ mới.
  \end{itemize}
\end{frame}

%------------------------------------------------------------------
\begin{frame}{Kiến trúc YOLO-World}
  \begin{center}
    \includegraphics[width=0.7\linewidth]{images/yolo-world-model-architecture-overview.png}
  \end{center}
  \vspace{0.2cm}
  \begin{itemize}
    \item Kết hợp mô hình thị giác với mô tả văn bản để phát hiện theo ngữ nghĩa.
    \item Cho phép mở rộng lớp PPE mới mà không cần huấn luyện lại từ đầu.
  \end{itemize}
\end{frame}

%------------------------------------------------------------------
\begin{frame}{Đánh giá \& hạn chế}
  \begin{itemize}
    \item Độ chính xác phụ thuộc chất lượng weights (\texttt{yolo8s\_ppedetect\_50e\_best.pt}); chưa có số liệu định lượng trong repo (cần chạy val lấy mAP).
    \item Zero-shot triển khai nhanh nhưng dễ nhầm nếu prompt không rõ hoặc mâu thuẫn (conf thấp).
    \item Chưa có tracking ID (multi-frame consistency), nên thống kê vi phạm mang tính frame-level.
    \item Logger chưa tách lưu vi phạm theo video/ca làm; chưa có dashboard.
  \end{itemize}
\end{frame}

%------------------------------------------------------------------
\begin{frame}{Hướng phát triển tương lai}
  \begin{itemize}
    \item Thêm \textbf{multi-object tracking} (ByteTrack/OC-SORT) để giảm duplicate và theo dõi từng công nhân.
    \item Bổ sung \textbf{asynchronous batch inference} cho video stream để tối ưu FPS trên CPU/GPU yếu.
    \item Mở rộng \textbf{rule engine} (per-zone, per-role, time-of-day) và UI config web.
    \item Xây dựng \textbf{dashboard} (FastAPI + React) cho real-time alert, thống kê vi phạm, truy xuất ảnh crop.
    \item Fine-tune thêm trên dữ liệu nội bộ (domain adaptation) và thử nghiệm \textbf{distillation} để nhẹ hơn.
  \end{itemize}
\end{frame}

%------------------------------------------------------------------
\begin{frame}{Tổng kết}
  \begin{itemize}
    \item Trọng tâm hiện tại là YOLOv8n đã huấn luyện, đảm bảo tốc độ và độ ổn định khi triển khai.
    \item Pipeline rõ ràng: camera/video $\rightarrow$ phát hiện $\rightarrow$ ghép thiết bị $\rightarrow$ kiểm tra quy tắc $\rightarrow$ ghi log/hiển thị.
    \item Dễ mở rộng nhờ cấu hình YAML và các module tách biệt.
    \item Hướng tương lai: YOLO-World để mở rộng PPE mới không cần gán nhãn, kèm tracking và dashboard.
  \end{itemize}
\end{frame}

\end{document}

